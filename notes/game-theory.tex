\documentclass[11pt]{scrartcl} 
\usepackage[sexy]{evan} % Taken from https://github.com/vEnhance/dotfiles/blob/main/texmf/tex/latex/evan/evan.sty

\begin{document}

\title{Game Theory}
\author{Timothy Mou}
\maketitle

\tableofcontents

\begin{abstract}
  These are some notes about the theory of impartial two-player games.
  The main focus is the game of Nim and the Sprague-Grundy theorem.
\end{abstract}

\section{Combinatorial Games}

\begin{definition}
  For our purposes, a \vocab{game} can be characterized by a set of states and transitions between the states, with the following properties:
  \begin{itemize}
    \item There are two players, sometimes called \textit{Left} and \textit{Right}, who take turns making moves (making a valid transition from one state to the next).
    Other than who moves first, the players are indistinguishable (i.e., they can make the same moves).
      This is a property of \textit{impartial} games.
   \item The players have perfect information, and the game is entirely deterministic.
   \item If a player cannot make a move (i.e., there are no valid transitions out of the current state), they are declared the \textit{loser}, and the other player is the \textit{winner}.
    \item The game must end after a finite sequence of moves.
      There will always be a winner and a loser; there can be no draws.
  \end{itemize}

  As a consequence, the outcome of a game with optimal play is completely determined by the current state.
  A state is winning if there is at least one transition to a losing state, and a state is losing if it has no transitions, or every transition is to a winning state.

  Because every game is finite, the game states can be modeled as a DAG, with directed edges representing transitions between states.
  We can use this run a simple DP starting from the sinks to compute whether a state is winning or losing.

  \subsection{Summing Games}


\end{definition}

\end{document}
