% !TEX TS-program = pdflatex
% !TEX encoding = UTF-8 Unicode

% This is a simple template for a LaTeX document using the "article" class.
% See "book", "report", "letter" for other types of document.

\documentclass[11pt]{article} % use larger type; default would be 10pt

\usepackage[utf8]{inputenc} % set input encoding (not needed with XeLaTeX)

%%% Examples of Article customizations
% These packages are optional, depending whether you want the features they provide.
% See the LaTeX Companion or other references for full information.

%%% PAGE DIMENSIONS
\usepackage{geometry} % to change the page dimensions
\geometry{a4paper} % or letterpaper (US) or a5paper or....
% \geometry{margin=2in} % for example, change the margins to 2 inches all round
% \geometry{landscape} % set up the page for landscape
%   read geometry.pdf for detailed page layout information

\usepackage{graphicx} % support the \includegraphics command and options

% \usepackage[parfill]{parskip} % Activate to begin paragraphs with an empty line rather than an indent

%%% PACKAGES
\usepackage{booktabs} % for much better looking tables
\usepackage{array} % for better arrays (eg matrices) in maths
\usepackage{paralist} % very flexible & customisable lists (eg. enumerate/itemize, etc.)
\usepackage{verbatim} % adds environment for commenting out blocks of text & for better verbatim
\usepackage{subfig} % make it possible to include more than one captioned figure/table in a single float
% These packages are all incorporated in the memoir class to one degree or another...

%%% HEADERS & FOOTERS
\usepackage{fancyhdr} % This should be set AFTER setting up the page geometry
\pagestyle{fancy} % options: empty , plain , fancy
\renewcommand{\headrulewidth}{0pt} % customise the layout...
\lhead{}\chead{}\rhead{}
\lfoot{}\cfoot{\thepage}\rfoot{}

%%% SECTION TITLE APPEARANCE
\usepackage{sectsty}
%\allsectionsfont{\sffamily\mdseries\upshape} % (See the fntguide.pdf for font help)
% (This matches ConTeXt defaults)

%%% ToC (table of contents) APPEARANCE
\usepackage[nottoc,notlof,notlot]{tocbibind} % Put the bibliography in the ToC
\usepackage[titles,subfigure]{tocloft} % Alter the style of the Table of Contents
\renewcommand{\cftsecfont}{\rmfamily\mdseries\upshape}
\renewcommand{\cftsecpagefont}{\rmfamily\mdseries\upshape} % No bold!

%%% END Article customizations

%%% The "real" document content comes below...

\title{USACO Platinum Preparation}
\author{Timothy Mou}
%\date{} % Activate to display a given date or no date (if empty),
         % otherwise the current date is printed 

\begin{document}
\maketitle

\section{Introduction}
Platinum is the highest tier of the USA Computing Olympiad, and 150-200 American high schoolers regularly compete in their four seasonal contests. A strong performance in these contests is crucial to being considered for USACO camp. 

The Platinum level poses a different challenge from the lower tiers of the competition. While the Gold level focuses on a narrow subset of techniques and data structures (mainly DP, BFS, DSU/minimal spanning trees), almost all competitive programming topics are fair game for the Platinum level. However, by looking at past contests, we can determine which topics come up most frequently and are the most important to study. They include:

\begin{itemize}
\item Segment trees (range queries, coordinate compression/line sweeps)
\item Advanced DP techniques
\item Tree techniques (centroid decomposition, DSU merge small-to-large trick, heavy-light decomposition)
\end{itemize}

These three topics (think of them as more like broad categories) have appeared regularly on Platinum contests, and I would consider these as the most important algorithms/data structures. Every contest will probably have at least one problem that falls into one of these three classifiers. If anything, they're a good place to start if you don't know what to study.  

Of course, there many other algorithms and techniques such as string algorithms, biconnected components, and square-root decomposition that may not appear frequently, but are expected to be well-known tools in any high level Platinum competitor's arsenal. In addition, there are countless algorithms that haven't ever appeared before that could be readily included in the next contest. The point is that just studying common algorithms will not be enough to regularly score high on contests. USACO problems are almost never straight-foward; they commonly require acute observations and strong implementation skills in addition to algorithmic knowledge. This motivates my proposal of a three-pronged plan to improve contest performance \footnote{Note: My plan is partly inspired by a lecture given during the 2018 UCF Competitive Programming camp. }. The areas of focus are:

\begin{enumerate}
\item Algorithms/data structure knowledge
\item Problem solving aptitude
\item Implementation and debugging skills
\end{enumerate}

Many Platinum competitors are strong at implementation and can quickly pick up new algorithms. The topic that is the most people struggle with and is hardest to train is problem solving skills. We will discuss these general areas in more detail.

\section{General improvement methods}

\subsection{Textbook Knowledge}














\end{document}
