\documentclass[12pt]
{article}

\topmargin0in
\headheight42pt
\oddsidemargin0in
\evensidemargin0in
\textwidth6.5in
\textheight8in

\usepackage{fancyhdr}
\usepackage{amsmath}
\usepackage{physics}
\usepackage{amssymb}
\usepackage{relsize}
\usepackage[shortlabels]{enumitem}
\usepackage{mathtools}
\DeclarePairedDelimiter\ceil{\lceil}{\rceil}
\DeclarePairedDelimiter\floor{\lfloor}{\rfloor}
\pagestyle{fancy}
\cfoot{}
\rhead{\bf
Timothy Mou \\
Assignment 3 \\
Page: \thepage}

\newcommand{\announce}[1]
{\vspace\baselineskip{\parindent0in {\bf #1}}}

\begin{document}
ch. 3: 34, 35, 41, 45-48, 53, 54

\announce{Problem 34.} 
Subgroups of $\bf{Z_3} \times \bf{Z_3}$: 
\begin{enumerate}
    \item $\{(0,0)\}$
    \item \{(0,0),(1,0),(2,0)\}
    \item \{(0,0),(0,1),(0,2)\}
    \item \{(0,0),(1,1),(2,2)\}
    \item \{(0,0),(1,2),(2,1)\}
    \item $\bf{Z_3} \times \bf{Z_3}$
\end{enumerate}

Subgroups of $\bf{Z_9}$:
\begin{enumerate}
	\item \{0\}
	\item \{0,3,6\}
	\item $\bf{Z_9}$
\end{enumerate}

Since $\bf{Z_3} \times \bf{Z_3}$ has a different number of subgroups than $\bf{Z_9}$, they must be different.
	
\announce{Problem 35.}
Let's label the elements of the symmetry group as $\{e,r_1,r_2,s_1,s_2,s_3\}$, where $r_1,r_2$ are rotations, and $s_1,s_2,s_3$ are reflections about one of the three sides. Then the subgroups are:
\begin{enumerate}
	\item $\{e\}$
	\item $\{e,r_1,r_2\}$
	\item $\{e,s_1\}$
	\item $\{e,s_2\}$
	\item $\{e,s_3\}$
	\item $\{e,r_1,r_2,s_1,s_2,s_3\}$
\end{enumerate}

\announce{Problem 41.}
The identity of $\bf{R^*}$ is $e = 1 + 0 \sqrt{2}$, so $e \in G$.
Let $A = a + b \sqrt{2} \in G$. 
Then since $A \neq 0 + 0\sqrt{2}$, $A^{-1} = \frac{1}{a + b\sqrt{2}} = \frac{1}{a + b\sqrt{2}} \cdot \frac{a-b\sqrt{2}}{a-b\sqrt{2}} = \frac{a-b\sqrt{2}}{a^2-2b^2} = \frac{a}{a^2-2b^2} + -\frac{b}{a^2-2b^2} \sqrt{2}$, and $\frac{a}{a^2-2b^2} , -\frac{b}{a^2-2b^2} \in \bf{Q}$, so $A^{-1} \in G$. 
If $h_1 = a + b\sqrt{2}$ and $h_2 = c + d\sqrt{2}$, then $h_1 h_2 = (a + b\sqrt{2})(c + d\sqrt{2}) = ac + 2bd + ad\sqrt{2} + b\sqrt{2}c = (ac + 2bd) + (ad + bc)\sqrt{2}$, which is in $G$. Therefore, $G$ is a subgroup of $\bf{R^*}$.

\announce{Problem 45.}
Let $H, K$ be subgroups of $G$. 
Since $e \in H$ and $e \in K$, $e \in H \cap K$. 
If an element $h \in H \cap K$, then $h^{-1} \in H$ and $h^{-1} \in K$, so $h^{-1} \in H \cap K$. 
If $h_1, h_2 \in H \cap K$, then $h_1 h_2 \in H$ and $h_1 h_2 \in K$, so $h_1 h_2 \in H \cap K$. 
Therefore $H \cap K$ is a subgroup of $G$.

\announce{Problem 46.}
Let $H = \{e,r_1,r_2\}$ and $K = \{e,s_1\}$ be subgroups of the symmetry group of the equilateral triangle. 
Then $H \cup K = \{e,r_1,r_2,s_1\}$ is not a subgroup, as $s_1r_1 \not\in H \cup K$.
Therefore the statement is false.

\announce{Problem 47.}
Let $H = \{e,s_1\}$ and $K = \{e,s_2\}$ be subgroups of the symmetry group of the equilateral triangle. 
Then $HK = \{e,s_1,s_2,s_1s_2\}$ is not a subgroup.
Therefore the statement is false.

In the case that $G$ is abelian, then $ee = e \in HK$. Let $a = hk, b = h'k' \in HK$, where $h,h' \in H, k,k' \in K$. Then $ab = hkh'k' = (hh')(kk') \in HK$, because $hh' \in H$ and $kk' \in K$. 
If $hk \in HK$, then $(hk)^{-1} = h^{-1}k^{-1} \in HK$, since $h^{-1} \in H$ and $k^{-1} \in K$. Therefore if $G$ is abelian, then $HK$ is a subgroup of $G$.

\announce{Problem 48.}
Clearly $e \in Z(G)$ since $ge = eg = g$ for all $g \in G$. If $x \in Z(G)$, then $gx =xg \iff x^{-1}gxx^{-1} = x^{-1} x g x^{-1} \iff x^{-1} g = g x^{-1}$, so $x^{-1} \in Z(G)$. 
If $x,y \in Z(G)$, then $gx = xg \implies g(xy) = x(gy) = (xy)g$, so $xy \in Z(G)$. Therefore $Z(G)$ is a subgroup of $G$.

\announce{Problem 53.}
$e \in C(H)$ since $eh = he$ for all $h \in H$. If $g \in C(H)$, then $gh = hg \iff g^{-1} g h g^{-1} = g^{-1} h g g^{-1} \iff h g^{-1} = g^{-1} h$, so $g^{-1} \in C(H)$. 
If $g_1, g_2 \in C(H)$, then $g_1 h = h g_1 \implies g_1 h g_2 = h (g_1 g_2) = (g_1 g_2) h$, so $g_1 g_2 \in C(H)$. Therefore $C(H)$ is a subgroup of $G$.

\announce{Problem 54.}
Since $e \in H$, $geg^{-1} = e$, so $e \in gHg^{-1}$. Let $h \in H$, and $ghg^{-1} \in gHg^{-1}$. Then $(ghg^{-1})^{-1} = (g(gh)^{-1})^{1} = gh^{-1}g^{-1}$. Since $h^{-1} \in H$, then $gh^{-1}g^{-1} \in gHg^{-1}$, so $(ghg^{-1})^{-1} \in gHg^{-1}$.
Let $gh_1g^{-1}, gh_2g^{-1} \in gHg^{-1}$. Then $(gh_1g^{-1})(gh_2g^{-1}) = gh_1(g^{-1}g)h_2g^{-1} = g(h_1h_2)g^{-1}$, which is in $gHg^{-1}$ since $h_1 h_2 \in H$. Therefore $gHg^{-1}$ is a subgroup of $G$.

\end{document}